\documentclass{article}
\usepackage[margin=1in]{geometry}
\usepackage{hyperref}

\title{ScanGuard AI: Context-Aware Static Analysis}
\author{Team CSIS}
\date{2025}

\begin{document}
\maketitle

\section*{Overview}
ScanGuard AI reduces static analysis noise by running Semgrep first and then using LLMs to interpret code context, filter false positives, and reprioritize findings based on exploitability.

\section*{Track Alignment}
\textbf{Track 1: AI-Enhanced DevSecOps Pipeline} \newline
Primary focus on context-aware SAST and exploitability scoring. \newline
\textbf{Track 4: Bug Triage Automation} \newline
Secondary focus on automated triage workflows for findings.

\section*{Architecture}
\begin{itemize}
  \item Repo fetcher for GitHub URLs and branches
  \item Semgrep scan with language-aware rulesets
  \item Context extraction around each finding
  \item LLM triage to mark false positives and adjust severity
  \item Deduplication and ranking via embeddings (Pinecone)
\end{itemize}

\section*{Impact}
\begin{itemize}
  \item Reduce false positives by 70\%+
  \item Prioritize exploitable issues first
  \item Shorten time-to-fix by focusing developer effort
\end{itemize}

\section*{Demo Flow}
\begin{enumerate}
  \item Trigger scan on a public repo
  \item Semgrep produces raw findings
  \item LLM triage filters noise and adjusts severity
  \item Dashboard displays the top real issues
\end{enumerate}

\section*{References}
\begin{itemize}
  \item \href{https://semgrep.dev/}{Semgrep Documentation}
  \item \href{https://www.pinecone.io/}{Pinecone Documentation}
\end{itemize}

\end{document}
